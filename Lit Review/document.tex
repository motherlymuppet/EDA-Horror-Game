\documentclass{article}

\bibliographystyle{style}

\usepackage{multicol}

\title{AI Search for Relaxing Music with Electrodermal Activity}
\author{Steven Lowes}
\date{5th October, 2018}

\setlength\parindent{0in} %indent and add lines between paras
\setlength{\parskip}{0.5\baselineskip}%


%packages to load last
\usepackage[hyphens, spaces]{url}%allow URLS
\usepackage{varioref}%Automatic page references
\usepackage[colorlinks, allcolors=blue, breaklinks]{hyperref}%Automatic reference links
\usepackage[all]{hypcap}
\usepackage[capitalise]{cleveref}%Automatic reference typing

\begin{document}
	\maketitle
	
	\section{Background}
	Music has the power to cause intense emotions in the listener \cite{agostino_how_2008}, and the power to completely relax them \cite{elliott_relaxing_2011}. Its power in manipulating emotions has led to widespread use in therapy, a field known as musical therapy \cite{american_music_therapy_association_definition_nodate}.
	
	The music used in music therapy is chosen specifically for the individual patient, respecting the fact that listeners have different responses to the same music \cite{american_music_therapy_association_definition_nodate}. This research aims to replace the need for a music therapy professional by implementing a system that dynamically searches for songs that relax an individual. To measure how relaxed the listener is, we can use an EDA sensor.
	
	\section{Terminology}
	\begin{itemize}
		\item \textbf{EDA} --- \emph{Electrodermal Activity}, also known as \emph{Skin Conductance} or \emph{Galvanic Skin Response}, is a measure of the subconscious sweating in the hands. It is caused by, and proportional to, stress in the sympathetic nervous system, which indicates intense emotion \cite{farnsworth_what_2018, boucsein_electrodermal_2012}.
		\item \textbf{Optimization Problem} --- A problem where the goal is to find the solution $x$ from some set of valid solutions $X$ which maximises/minimises some \emph{Fitness Function} $f$ \cite{boyd_convex_2004}. Our problem of finding the most relaxing song is an optimization problem.
		\item \textbf{Cluster Analysis} --- Also known as \emph{Clustering}, is the process of dividing a large set into multiple smaller sets that share some property \cite{estivill-castro_why_2002}. We will talk specifically about \emph{hard k-means clustering}, a computation that groups a set of objects based on their distance from the mean of the cluster \cite{kanungo_efficient_2002}.
		\item \textbf{Surrogate} --- A \emph{Surrogate Function} is a function which approximates a more complex function while being much more convenient to run \cite{mairal_optimization_2013}.
		\item \textbf{EBOP} --- An \emph{Expensive Black Box Optimization Problem} is an optimization problem where each call to the fitness function is \emph{expensive} (in our case slow) and where we cannot use domain knowledge to guide the search \cite{jones_efficient_1998}. These problems are common in Engineering.
	\end{itemize}
	
	\section{Proposed Solution}
	Using the Spotify API, we can represent songs as a 14-dimensional vectors which describe their important qualities, including \emph{danceability}, \emph{energy}, and \emph{musical key} \cite{spotify_get_nodate, jehan_analyzer_nodate}. We have a database of approximately 500,000 songs on Spotify, and the ability to search them based on their vector representations.
	
	We model the problem as a Global Optimization Problem. The set of solutions are the 500,000 songs in the database. To determine the fitness of any solution, we play it to a listener, and measure the change in EDA. A good solution should lower the listener's EDA.
	
	\subsection{Assumptions}
	The following assumptions have been made and will need to be tested and confirmed before the solution can be implemented:
	
	\begin{itemize}
		\item The fitness of one solution is roughly the same when it is tested twice (there is little error in the fitness function)
		\item Solutions which have similar vectors usually have similar fitness
		\item A song which reduces EDA in the listener correlates with the listener self-reporting feeling relaxed
	\end{itemize}

	\subsection{Areas of Concern}
	Testing the fitness of one solution will require listening to an entire song. This means each invocation of the fitness function will take approximately 3 minutes. This results in an upper bound of 100 calls to  the fitness function per search, equivalent to 5 hours of music listening. Since we want to represent differences in preference between listeners, we cannot run the fitness function in parallel with multiple listeners.

	There are some mitigating factors, which reduce the chance that time constraints will make this solution unviable:
	
	\begin{itemize}
		\item We know that two solutions which are similar have similar fitness, so we can reduce the number of invocations required by performing clustering analysis.
		\item We have essentially unlimited computation power for anything that does not involve a human listener. Even computationally complex analysis will run quickly with only 500,000 solutions to test.
		\item We may be able to play only some of a song to test its fitness. This would speed up execution. The Spotify API can provide 30-second previews of songs \cite{spotify_get_nodate-1}, which usually align with the chorus of the song.
	\end{itemize}

	\section{Literature Survey}
	
	\subsection{Music as Vectors}
	The Spotify API allows us to convert songs to vectors \cite{spotify_get_nodate}. The Echo Nest documentation gives more details on what each of the qualities used in the vector represents. Since we are optimizing without using any domain knowledge, this information is mostly irrelevant for us. The qualities of a song are determined using `machine listening techniques' guided by music experts \cite{jehan_analyzer_nodate}.
	
	Other work that has done clustering analysis has found that the clusters mostly match up with what we expect (genres), which suggests one of our assumptions (similar vectors = similar songs = similar fitness) is true \cite{santos_discovering_2017, rcharlie_analytics_coachellar_2017}.
	
	\subsection{Electrodermal Activity}
	EDA will be measured with a Grove GSR Sensor \cite{seeedstudio_grove_nodate}, which can be read into a PC via an arduino. Similar sensors are used for detection of stress by polygraph (lie-detector) systems \cite{tuckett_detection_1986}. Listening to music has been shown to cause an increase in EDA \cite{rickard_intense_2004, dillman-capentier_effects_2007, thoma_effect_2013}.
	
	\subsection{Global Optimization Search}
	Our fitness function consists of playing a song to a listener and measuring their change in EDA. This means that each invocation of the fitness function takes approximately 3 minutes. The fitness function cannot be calculated in parallel, due to the need for a human listener, and we cannot use multiple human listeners due to the differences in individual preference that we are testing.
	
	Multiple state-of-the-art AI Search techniques were considered, but all were deemed to be too time-complex for our use case. These included Tabu Search, Hill Climbing, Simulated Annealing, Genetic Algorithms, Newton's Algorithm, and the Bee Algorithm \cite{chandel_searching_2014, pham_bee_2005}.
	
	It is clear that we need to reduce the time complexity of our search, even at the cost of a very non-optimal solution. A StackExchange user lead me to research into the solving of EBOPs \cite{stackexchange_search_2018}.
	
	These EBOP-solving algorithms use methods such as \emph{clustering} and \emph{surrogates} to reduce the number of times we need to invoke the expensive fitness function \cite{boukouvala_derivative-free_2014, jones_efficient_1998, dong_surrogate-based_2018}.
	
	\subsection{Clustering}
	Clustering can be used to reduce the number of function invocations needed in our search \cite{boukouvala_derivative-free_2014, wang_time_2017}. Clustering our solutions means that we can reduce the number of solutions without reducing the amount of diversity in the set of solutions.
	
	Clustering is computationally resource-intensive. However, we are not concerned about reducing the amount of computation done, as the cost of listening to songs (fitness function invocations) is far higher than the cost of clustering.
	
	\subsection{Bayesian \& Surrogate Techniques}
	These techniques work by constructing a surrogate model of what we predict the fitness function to be, and iteratively improve on that surrogate each time more data becomes available \cite{mockus_optimization_nodate, dong_surrogate-based_2018}. By maintaining a surrogate of the fitness function, we can perform a search on the surrogate and test only the solutions that are predicted to be globally optimal. This technique reduces the number of calls to the fitness function by maximising the impact of the tested solutions, at the cost of having to construct a new surrogate and perform a global search on the surrogate after each call to the real fitness function \cite{boukouvala_derivative-free_2014}.
	
	\section{Conclusion}
	We aim to use Electrodermal Activity to determine how relaxed a listener is. The data from the EDA sensor can be used to guide an Optimization search across a space of 500,000 songs to find the most relaxing song. To calculate a song's fitness, we must play the song to a listener, which can take many minutes. This means the search is categorised as an Expensive black-box Optimization Problem. Techniques such as Cluster Analysis and Surrogate Functions can be used to minimise the number of calls to the expensive fitness function in our search at the cost of additional computational complexity.
	
	\newpage
	\begin{footnotesize}
		\begin{multicols}{2}
			\bibliography{document}
		\end{multicols}
	\end{footnotesize}
	
\end{document}