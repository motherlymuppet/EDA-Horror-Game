\documentclass{article}

\usepackage{tabu}
\usepackage{hyperref}

\title{{\Huge 3\textsuperscript{rd} Year Project Log Book}}
\author{Steven Lowes}
\date{Last Updated \textbf{\today}}

\setlength\parindent{0in} %indent and add lines between paras
\setlength{\parskip}{\baselineskip}%

\begin{document}
	\maketitle
	
	\newpage
	
	\tableofcontents
	
	\newpage
	
	\section{Pre-Log}
	Brief overview of what had been done before this log began:
	
	Started off wanting to generate music with EEG headset to invoke emotions using machine learning. Asked around with PhD students, they didn't really think it was possible. I emailed Toby about doing that project with him, and he passed me off to Magnus. I emailed Magnus and after a bit of back and forth we decided to simplify our input by just using GSR instead of EEG and by using pre-made songs on Spotify.
	
	\section{2018-06-29}
	I met with Magnus, having created a first draft of my Project Plan. We discussed ethics forms, equipment purchase, and the experimental design of the project. We agreed that a first stage of the project should involve manually selecting songs and playing them to listeners, measuring their GSR, and asking them to rate the songs in some way.
	
	This would allow us to demonstrate a correlation and at least show that GSR is measuring something worthwhile and is not just random noise.
	
	The ethics forms look relatively simple, and will just require me to think about the format of the experiments, and how to maintain anonymity. The anonymity is pretty easy, as the only thing that needs to published is GSR data, which is not PII (I think, I will double check that people don't have a personally identifiable fingerprint in their GSR data).
	
	I need to conduct a literature review before designing this experiment, and should probably experiment on myself before designing the experiment procedure and submitting for ethics.
	
	I was also told that I needed to keep a logbook. Here it is!
	
	As for equipment purchases, there's a budget of \pounds100. Magnus suggested getting multiple sensors, which I will definitely try to do. There is the issue of how I will read the GPIO pins into the PC, which would be helpful for the second stage of experimentation, but not necessary for the first. In fact, it might be easier to just use the raspberry pi as a standalone computer and have all the heavy lifting done on a separate computer / in the cloud and use a web api. I will look into that.
	
	\section{2018-06-30}
	I'm looking into equipment purchasing. There are no cheap GSR sensors that read directly into the PC. They are \~\pounds200 instead of \pounds10.
	
	In the end I went for this:
	
	\begin{tabu}{XXX}
		\textbf{Item}&\textbf{Price}&\textbf{Quantity}\\
		Raspberry Pi Zero WH&\pounds12&2\\
		\hyperlink{https://brownsestablishment.co.uk/Page_i1675597}{Grove GSR Sensor}&\pounds10&2\\
		\hyperlink{https://www.amazon.co.uk/Adapter-Upgraded-VicTsing-Gold-Plated-Converter-Black-1080p-Audio/dp/B016I34422/}{Audio breakout}&\pounds8&2\\
		\hyperlink{https://www.amazon.co.uk/Rankie-Adapter-smartphone-tablets-connection-black/dp/B00YOX4JU6/}{OTG Adapter * 3}&\pounds4&1\\
	\end{tabu}

	Edit - No i didn't. Full email transcript to Magnus:
	
	\begin{quote}
	Hi Magnus,
	
	The GSR sensor is available from various places but most seem a bit sketchy - it's also available from mouser, so i've used a link to that one. I can find it from other stores if that poses a problem.
	
	I looked into using a raspberry pi, but the lack of analog inputs will pose a problem for reading from the sensor. I've figured out how to use serial over usb to communicate with the arduino so that shouldn't be an issue - I can run the majority of the code on a laptop and just have the arduino constantly send the sensor readings over the serial connection.
	
	If possible, I would like 2 sensors so that I can experiment on two participants at once - or in case one of the sensors breaks for whatever reason. I'd still only need 1 arduino.
	
	In summary, this is the equipment request:
	
	1x \hyperlink{https://www.mouser.co.uk/ProductDetail/Seeed-Studio/110990210?qs=sGAEpiMZZMvxSQPygxWTpU\%252buE\%2fLKoESoNY3o\%252bR6\%2feGg=}{Jumper Wires}
	
	2x \hyperlink{https://www.mouser.co.uk/ProductDetail/Seeed-Studio/101020052?qs=\%2fha2pyFaduglX1VjfWhqo3FSmM\%252bgGP5Nj9\%2f\%2fuNr2Jy3k41ptw01cgA\%3d\%3d}{GSR Sensor}
	
	1x \hyperlink{http://uk.farnell.com/arduino/a000073/arduino-uno-smd-dev-kit/dp/2285200}{Arduino Uno}
	
	Thanks,
	Steven
	\end{quote}

	That equipment has been ordered.

	\section{2018-10-01}
	I'm working on my Lit Review today. I have emailed Magnus and got an extension until this Friday (5th) Due to my internship sucking up so much of my time. I've got the Lit Review brief off duo and am reading Karl's lit review too, for ideas.
	
	\section{2018-10-11}
	My lit review went ok. I spent a lot of time looking at how to more efficiently search the research space. I met Magnus today to discuss it but he didn't have any feedback to give me. The equipment had arrived though. I'll take that home and play with that for a bit.
	
	\section{2018-10-18}
	I've written up my ethics form and am having a meeting with Magnus tomorrow to go over than and get up to date with where the project is. There's a project plan coming up so I need to be thinking about how I intend for the project to progress.
	
	To break things down, this is how far we are with the implementation:
	
	\begin{itemize}
		\item Can read from the sensor into a java program
		\begin{itemize}
			\item The random noise from the sensor is approx $\pm5$ units
			\item The noise from distractions, moving, deep breaths, and other natural places is around $\pm50$ units.
			\item Intense stress responses can cause changes of $\pm1000$ units
			\item It takes approx 0.5s for a stress response to show up in sensor readings
			\item The readings stabilise at a new normal within about 2 seconds of a response
			\item It can take minutes for the readings to recover back to where they were, even if the listener has been as relaxed as they were before for almost all of it.
		\end{itemize}
		\item Can represent songs as vectors
		\begin{itemize}
			\item Spotify api is working, but it's not giving great results. Songs are similar in terms of energy, but sound very different. E.g. the bottom song is the closest it could find to the top song.
			
			\begin{itemize}
				\item \hyperlink{ID 3E6Ydoy1H8ePoigc10Mgsa}{https://p.scdn.co/mp3-preview/477ea14e861597aec13fee51e609e88c9d85d50a?cid=f9960a1f12f04266be8c080100a137a6}
				\item \hyperlink{ID 3KNQFdsXrTvG9CF3BioxIa}{https://p.scdn.co/mp3-preview/5749007183d53ccec68c1f57d8bad3eb6030b196?cid=f9960a1f12f04266be8c080100a137a6}
			\end{itemize}
		
			This isn't a joke, and I didn't specifically pick that first songs. These 2 songs have:
			
			\begin{itemize}
				\item \textbf{Identical} Key (C Major)
				\item \textbf{Identical} Time Signature (3/4)
				\item \textbf{Within 15\%} Danceability, Energy,
			\end{itemize}
			
			\item If I use the api to download a subset of all spotify songs' data, I can use the advanced analysis endpoint which will allow me to create vectors that more closely represent the song. (Timbre)
		\end{itemize}
	\end{itemize}

	There are still some questions that need answering:

	\begin{itemize}
		\item How can we tell how stressed someone was over a period of time if the recovery takes much longer than the initial stress response?
		\item How can we better represent songs as vectors
		\item Do we need to do a better job, or is the current representation good enough?
		\item Are short clips representative of long clips?
	\end{itemize}

	\section{2018-10-22}
	Had another meeting with Magnus. Need to do a project plan. Ethics is approved.
	
	Spent most of my time working on representing songs as vectors. It's got waaay better now. In albums where the songs are designed to follow on from each other, it will often pick the song before/after the one we search for. That's awesome! I haven't seen anything that's glaringly wrong so far.
	
	Currently using a 500k song dataset, but slowly downloading the full 25m song dataset. That script is running on my server at home but will take about a month to complete fully. Thankfully I can pull a partial dataset at any point and use it.
	
	I'm doing an open mic night in 3 days about my diss, so I need something to present. I want to get data saving done before then, and start capturing some data.
	
	\section{2018-10-25}
	Music as vectors is still coming along nicely. After changing the vectors to be a vector of z-scores, it improved a lot. I think i'm happy with it for now, even though it's nowhere near perfect. The song database download is chugging along and will take another 3 weeks or so.
	
	My diss talk is today, and I don't have much to present in terms of a demo, which is sad. It's mostly a theoretical talk and background. Slides are available here: \url{https://docs.google.com/presentation/d/1VicKOVZO1Z2ydYGs1vnxBoXv69i877GCZx32zt7ANLA/edit?usp=sharing}
	
	I haven't done much of what Magnus asked, but I'm going to get the project plan done, and get some charts to show him. I need an easy way to export chart data and then re-load it so it can be viewed. For now I think i'll just have to mess around for a bit and get some screenshots and annotate them.
	
	\section{2018-10-28}
	I've spent the weekend working on the GUi, using TornadoFX. It's been slow progress due to learning a new graphics framework but i'm getting pretty good now and getting good results. The GUI for data recording is done, and allows you to save a screenshot of the collected data.
	
	I also encode all of the recorded data into the comment tEXt field of the PNG. This will mean that I can reload the chart for viewing by loading the png into the program. I will need to consider how to get across to users that this is possible.
		
\end{document}