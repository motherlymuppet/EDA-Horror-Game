
\documentclass[a4paper,12pt]{article}

\bibliographystyle{apacite}

\usepackage[natbibapa, sectionbib, notocbib]{apacite}%Citations

%packages to load last
\usepackage[hyphens, spaces]{url}%allow URLS
\usepackage{varioref}%Automatic page references
\usepackage[colorlinks, allcolors=blue, breaklinks]{hyperref}%Automatic reference links
\usepackage[all]{hypcap}
\usepackage[capitalise]{cleveref}%Automatic reference typing
\usepackage{etoolbox} %lets us do all that funky stuff later on
\usepackage{regexpatch}

\setlength\parindent{0in} %indent and add lines between paras
\setlength{\parskip}{0.5\baselineskip}%

\makeatletter
\xpatchcmd{\@@cite}{\def\BCA##1##2{{\@BAstyle ##1}}}{\def\BCA##1##2{{\@BAstyle ##2}}}{}{}

% count citations
\pretocmd{\NAT@citex}{%
	\let\NAT@hyper@\NAT@hyper@citex
	\def\NAT@postnote{#2}%
	\setcounter{NAT@total@cites}{0}%
	\setcounter{NAT@count@cites}{0}%
	\forcsvlist{\stepcounter{NAT@total@cites}\@gobble}{#3}}{}{}
\newcounter{NAT@total@cites}
\newcounter{NAT@count@cites}
\def\NAT@postnote{}

% include postnote and \citet closing bracket in hyperlink
\def\NAT@hyper@citex#1{%
	\stepcounter{NAT@count@cites}%
	\hyper@natlinkstart{\@citeb\@extra@b@citeb}#1%
	\ifnumequal{\value{NAT@count@cites}}{\value{NAT@total@cites}}
	{\if*\NAT@postnote*\else\NAT@cmt\NAT@postnote\global\def\NAT@postnote{}\fi}{}%
	\ifNAT@swa\else\if\relax\NAT@date\relax
	\else\NAT@@close\global\let\NAT@nm\@empty\fi\fi% avoid compact citations
	\hyper@natlinkend}
\renewcommand\hyper@natlinkbreak[2]{#1}

% avoid extraneous postnotes, closing brackets
\patchcmd{\NAT@citex}
{\ifNAT@swa\else\if*#2*\else\NAT@cmt#2\fi
	\if\relax\NAT@date\relax\else\NAT@@close\fi\fi}{}{}{}
\patchcmd{\NAT@citex}
{\if\relax\NAT@date\relax\NAT@def@citea\else\NAT@def@citea@close\fi}
{\if\relax\NAT@date\relax\NAT@def@citea\else\NAT@def@citea@space\fi}{}{}
\patchcmd{\NAT@cite}{\if*#3*}{\if*\NAT@postnote*}{}{}

% all punctuation black
\AtBeginDocument{%
	\preto\NAT@sep{\textcolor{black}\bgroup}%
	\appto\NAT@sep{\egroup}%
	\preto\NAT@aysep{\textcolor{black}\bgroup}%
	\appto\NAT@aysep{\egroup}%
	\preto\NAT@yrsep{\textcolor{black}\bgroup}%
	\appto\NAT@yrsep{\egroup}%
	\preto\NAT@cmt{\textcolor{black}\bgroup}%
	\appto\NAT@cmt{\egroup}%
	\preto\NAT@open{\textcolor{black}\bgroup}%
	\appto\NAT@open{\egroup}%
	\preto\NAT@close{\textcolor{black}\bgroup}%
	\appto\NAT@close{\egroup}}
\makeatother

\pagenumbering{gobble}

\tolerance=1 %don't use hyphens
\emergencystretch=\maxdimen
\hyphenpenalty=10000
\hbadness=10000


\begin{document}
	{\Large\textbf{Research Question:}\\AI Search for Physiologically Arousing Songs}
	
	\textbf{Student Name:} Steven Lowes
	
	\textbf{Supervisor Name:} Magnus Bordewich
	
	\textbf{Project type:} 3\textsuperscript{rd} Year Project for BSc Computer Science
	
	\textbf{Subjects:}
	\begin{itemize}
		\item AI Search
		\item Sound and Music Computing
		\item Human-Computer Interaction
		\item Wearable Tech
	\end{itemize}
	
	\textbf{Objectives:}
	\begin{itemize}
		\item Effectively search a large space quickly when computing the fitness value of a point takes minutes.
		\item Listeners should feel an increase in physiological arousal over time while the search is active and when replaying the songs which caused the largest increases during the search.
		\item Listeners should intuitively see similarities between songs that are classified as highly arousing, and differences between those songs and songs classified as relaxing.
		\item Visually show the differences in what music arouses multiple listeners.
	\end{itemize}
	
	\textbf{Deliverables:}
	\begin{itemize}
		\item \textbf{Minimum}
		\begin{itemize}
			\item Read user's physiological arousal levels into a computer
			\item Algorithm that searches the space for highly arousing songs
			\item Integration with the Spotify API to fully automate the search
		\end{itemize}
		\item \textbf{Intermediate}
		\begin{itemize}
			\item Produce graphical representation of fitness landscape
			\item Automatically populate playlists with highly arousing songs for later listening
			\item Test the arousing songs again at a later date to determine whether arousal is stable in that songs that arouse you one day will do so in a week's time also
			\item Try to minimise arousal instead of maximising it to find relaxing songs
		\end{itemize}
		\item \textbf{Advanced}
		\begin{itemize}
			\item Make it into a wearable device
			\item Explore using alternative sensors, such as heart rate monitor or simple EEG headset to monitor focus levels.
		\end{itemize}
	\end{itemize}

	\section{Background}
	
	Sound and music can cause intense emotional reaction in listeners \citep{arousingMusic}, but the action of storing music in a playlist can easily distract the listener and ruin the experience. In addition to this, each listener may respond differently to the same song based on context, past experiences, and psychology \citep{musicDifferences}.
	
	Physiological arousal is the state of the brain and sympathetic nervous system being `awake', or highly active, often due to intense emotion or emotional stress. When a song causes a listener to get `pumped up', it is accompanied by an increase in the user's physiological arousal \citep{pumpedUp}. This can be measured using galvanic skin response (GSR) sensors, which measure a change in arousal \citep{GSR}. These are commonly used in lie detectors and are available very cheaply \citep{gsrSensor}.
	
	%\section{Preliminary Preparations}
	
	\section{Discussion of how to Achieve Objectives}
	
	We can model the problem of finding a song which causes increased physiological arousal as an AI search problem. The Spotify Web API includes an endpoint named \verb|audio-features|, which returns a list of 13 scalars describing the properties of the song \citep{audioFeatures}. These scalars include `danceability', `loudness', and `instrumentalness', among others, and can be used as the axes of a high-dimensional space. We can map a fitness landscape onto this space, where each song is represented by a point in the space, and each point in the space has a fitness determined by how much the song increased a listener's physiological arousal.

	The main difficulty in this project is how to perform a robust search of the fitness landscape with very limited data. It will take the full length of each song to determine the listener's change in arousal, and the task cannot be easily parallelised due to the need for a human listener, and the fact that each listener may respond differently to each song.
	
	Once this system is working reliably, we will develop it into a wearable device so it can used portably. This system has applications for the general consumer, in addition to potential uses in a music therapy setting \citep{musicTherapy}. Further research in the area could move to using more complex sensors, such as EEG headsets, rather than GSR sensors, or alternatively generating physiologically arousing music from scratch.
	
	\section{Timeline}
	
	\begin{itemize}
		\item \textbf{Summer Break:}
		\begin{itemize}
			\item Get Ethics Approval
			\item Order Equipment
			\item Do background reading on AI Search and Physiological Arousal
			\item Set up equipment, measuring arousal and reading into program
		\end{itemize}
		\item \textbf{1st Term:}
		\begin{itemize}
			\item Test songs, understand baseline and gain an intuitive understanding of what kind of songs increase arousal in myself
			\item Start writing + make good progress with AI Search algorithm
		\end{itemize}
		\item \textbf{2nd Term:}
		\begin{itemize}
			\item Test on other listeners
			\item Compare the fitness landscapes of multiple listeners
			\item Experiment with fitness function, i.e. finding songs that decrease arousal
		\end{itemize}
		\item \textbf{Easter Break:}
		\begin{itemize}
			\item Write report
		\end{itemize}
	\end{itemize}
	
	\bibliography{document}
\end{document}